\chapter{Conclusiones}

\label{Chapter5}

En el presente capítulo se detallan los resultados obtenidos del trabajo realizado y se describen las mejoras en un posible trabajo futuro.

\section{Resultados obtenidos}

En este trabajo se concluyó el desarrollo y pruebas funcionales de un prototipo de sistema de automatización de hogares, contando con 2 tipos de dispositivos.

Para evaluar los resultados finales, es necesario analizar los siguientes temas:

\begin{itemize}
	\item Se cumplió con las fechas finales de finalización del trabajo, aunque la realización de las tareas en la práctica difirió un poco de lo planificado. Se planificaron las tareas de forma muy secuencial pero se hicieron algunas de ellas en paralelo junto con otras.
	\item No se manifestó ninguno de los riesgos advertidos en la planificación, aunque se siguieron todas las medidas de mitigación de aquellos riesgos más graves.
	\item Se lograron cumplir con todos los requerimientos establecidos y acordados con el cliente, aunque fue necesario realizar una modificación que implicó dividir el nodo previsto, de modo que ahora dos nodos tengan funciones específicas.
\end{itemize}

Fueron de gran aporte y utilidad todos los conocimientos adquiridos en el transcurso del posgrado, y en especial aquellos conocimientos adquiridos en las materias que se listan a continuación:

\begin{itemize}
	\item Protocolos de Internet.
	\item Arquitecturas de datos.
	\item Arquitecturas de protocolos.
	\item Desarrollo de aplicaciones multiplataforma.
	\item Desarrollo de aplicaciones para Internet de las cosas.
\end{itemize}

\section{Trabajo futuro}

A continuación, se describen las tareas necesarias para implementar mejoras en el sistema, y su ejecución se contemplará en un posible trabajo futuro:

\begin{itemize}
	\item Hacer aplicaciones móviles para Android y iOS, ya que el desarrollo fue hecho en Ionic con este propósito.
	\item Implementar un \textit{access point} en el servidor para que los dispositivos se conecten automáticamente a él una vez que sean encendidos por primera vez.
	\item Implementar actualizaciones de software OTA en los dispositivos.
	\item Integración con sistemas de seguridad y cámaras.
	\item Mejoras en la interfaz gráfica de la web tales como el uso de menúes en las páginas en lugar de usar solo botones.
	\item Integración con Google Assistant.
	\item Implementación de un servicio en la nube para el acceso fuera del hogar o edificación, manteniendo todas las otras funcionalidades en el servidor local.
	\item Implementación de niveles de usuarios para que haya un usuario \textit{root} que pueda crear cuantos usuarios desee y los administre.
	\item Implementación de alarmas y avisos a través de la aplicación móvil o web y a través de plataformas como Telegram.
	\item Diseño de circuitos impresos y gabinetes para los nodos.
\end{itemize}